\documentclass[letter,twoside,12pt]{article}
\usepackage{lmodern}
\usepackage[T1]{fontenc}
\usepackage[spanish]{babel}
\usepackage[utf8]{inputenc}
\usepackage{amsmath}
\usepackage{amssymb}
\usepackage{amsthm}
\usepackage{amsthm}
\usepackage{fullpage}
\usepackage{latexsym}
\usepackage{enumerate}
\usepackage{enumitem}
\usepackage{braket}
\PassOptionsToPackage{hyphens}{url}\usepackage{hyperref}
\title{Computación Cuántica: Tarea \#1}
\newtheorem{theo}{Teorema}
\newtheorem{lemma}[theo]{Lema}
\newtheorem*{defi}{Definición}
\author{Jonathan Andrés Niño Cortés}
\begin{document}
\maketitle
\begin{enumerate}
\item Si
\begin{equation}
|a \rangle = 
\begin{pmatrix}
-2
\\ 4i
\\ 1
\end{pmatrix} \,
|b \rangle = 
\begin{pmatrix}
1
\\ 0
\\ i
\end{pmatrix} \nonumber
\end{equation}
\begin{enumerate}
\item $ \langle a | = \begin{pmatrix}
-2 & -4i & 1
\end{pmatrix} $

\item $ \langle b | = \begin{pmatrix}
1 & 0 & -i
\end{pmatrix} $

\item $ \langle a | b \rangle = -2+ i$

\item $ \langle b | a \rangle = -2 -i $

\item $ | c \rangle = | a \rangle +2| b \rangle = \begin{pmatrix}
0 
\\ 4i 
\\ 1+2i
\end{pmatrix} $

\item $ \langle c | a \rangle = \begin{pmatrix}
-15-2i
\end{pmatrix} $

\item $ \frac{| a \rangle}{||a||} = \frac{1}{\sqrt{21}}\begin{pmatrix}
-2 
\\ 4i 
\\ 1
\end{pmatrix}  $
\end{enumerate}
\item 

$ |+ \rangle = \frac{|0 \rangle + |1 \rangle}{\sqrt{2}} = \frac{1}{\sqrt{2}}
\begin{pmatrix}
1
\\1
\end{pmatrix} $ y $ |- \rangle = \frac{|0 \rangle - |1 \rangle}{\sqrt{2}} = \frac{1}{\sqrt{2}}
\begin{pmatrix}
1
\\-1
\end{pmatrix} $

La matriz de cambio de base de esta base a la canónica es
$ \frac{1}{\sqrt{2}}\begin{pmatrix}
1 &  1
\\ 1 & -1
\end{pmatrix}
 $
 
Por otra parte la matriz inversa obtenida a través de Gauss-Jordan da de nuevo $ \frac{1}{\sqrt{2}}\begin{pmatrix}
1 &  1
\\ 1 & -1
\end{pmatrix}
 $. Esta es la propiedad que esperabamos de la compuerta de Hadamard.
\item Utilizando Grand-Schmidt, tome $ |w_1 \rangle = \frac{v_1}{||v_1||} = \frac{1}{\sqrt{2}}\begin{pmatrix}
i
\\ 1
\end{pmatrix} $.

Ahora tome $ |w_2' \rangle = |v_2 \rangle - \frac{\langle v_1 | v_2 \rangle}{\langle v_1 | v_1 \rangle} | v_1 \rangle = \begin{pmatrix}
1+i 
\\ 2
\end{pmatrix}-\frac{3-i}{2}\begin{pmatrix}
i 
\\ 1
\end{pmatrix} = \begin{pmatrix}
1+i-\frac{1+3i}{2} 
\\ 2-\frac{3-i}{2}
\end{pmatrix} = \begin{pmatrix}
\frac{1-i}{2} 
\\ \frac{1+i}{2}
\end{pmatrix}$

Y $ |w_2 \rangle = \frac{|w_2 ' \rangle}{\langle w_2' | w_2' \rangle} = |w_2' \rangle$ pues, $ \langle w_2' | w_2' \rangle = 1$.  

\item Sea $ |h \rangle $ y $ |v \rangle $ vectores ortogonales y unitarios. Sea
\begin{eqnarray}
|\psi_1 \rangle &=& 1/2|h \rangle + \sqrt{3}/2 | v \rangle \nonumber
\\|\psi_2 \rangle &=& 1/2|h \rangle - \sqrt{3}/2 | v \rangle \nonumber
\\|\psi_3 \rangle &=& |h \rangle \nonumber
\end{eqnarray} 
Calcular 
\begin{enumerate}
\item $ \langle \psi_1 | \psi_2 \rangle = \langle \psi_1 | (1/2|h \rangle - \sqrt{3}/2 | v \rangle) \rangle = 1/2 \langle \psi_1 | h \rangle - \sqrt{3}/2\langle \psi_1 | v \rangle $

$ =  1/2 \langle 1/2|h \rangle + \sqrt{3}/2 | v \rangle | h \rangle - \sqrt{3}/2\langle 1/2|h \rangle + \sqrt{3}/2 | v \rangle | v \rangle $

$ = (1/2)(1/2) \langle h|h \rangle + (\sqrt{3}/2)(1/2) \langle v| h \rangle  - (\sqrt{3}/2)(1/2)\langle h |v \rangle - (\sqrt{3}/2) (\sqrt{3}/2) \langle v | v \rangle $

$ = 1/4 + 0 + 0 - 3/4 = - 1/2 $.

Luego $ |\langle \psi_1|\psi_2\rangle|^2 = 1/4 $

\item $ \langle \psi_1 | \psi_3 \rangle = \langle \psi_1 | h \rangle = \langle 1/2|h \rangle + \sqrt{3}/2 | v \rangle | h \rangle = 1/2 \langle h|h \rangle + \sqrt{3}/2 \langle v|h \rangle = 1/2$

Luego $ |\langle \psi_1|\psi_3\rangle|^2 = 1/4 $

\item $ \langle \psi_3 | \psi_2 \rangle = \langle h | \psi_2 \rangle = \langle h | 1/2|h \rangle - \sqrt{3}/2 \ = 1/2\langle h |h \rangle - \sqrt{3}/2 \langle h |v \rangle = 1/2 $

Luego $ |\langle \psi_3|\psi_2\rangle|^2 = 1/4 $

\end{enumerate}

\item \begin{enumerate} \item Por el hint podemos ver que $ |0 \rangle \langle 0| $ es equivalente a proyección sobre el espacio generado por $ |0 \rangle $. Asimismo, $ |1 \rangle \langle 1 | $ es la matriz de proyección en la segunda coordenada. Luego,

$ |0 \rangle \langle 0| - |1 \rangle \langle 1| = \begin{pmatrix}
1 & 0
\\ 0 & -1
\end{pmatrix} $
\item $ |00 \rangle \langle 11|$ es el operador que envia $ |x \rangle $ a $ \langle 11|x \rangle |00 \rangle $ entonces la matriz que la representa es

$$ \begin{pmatrix}
0&0&0&1
\\0&0&0&0
\\0&0&0&0
\\0&0&0&0
\end{pmatrix} $$

Los demás son similares y dan como resultado
$ |00\rangle \langle 11|+2|00 \rangle  \langle 10|+3|11 \rangle \langle 01| = $

$ \begin{pmatrix}
0&0&0&1
\\0&0&0&0
\\0&0&0&0
\\0&0&0&0
\end{pmatrix} $ +
$ \begin{pmatrix}
0&0&2&0
\\0&0&0&0
\\0&0&0&0
\\0&0&0&0
\end{pmatrix} $+
$ \begin{pmatrix}
0&0&0&0
\\0&0&0&0
\\0&0&0&0
\\0&3&0&0
\end{pmatrix} =
\begin{pmatrix}
0&0&2&1
\\0&0&0&0
\\0&0&0&0
\\0&3&0&0
\end{pmatrix}$

\end{enumerate}

\item Vamos a demostrar que $ A = \sum_{i=1}^n | u_i \rangle \langle u_1 |$ es la identidad. Tome cualquier vector $ |x \rangle $ en el espacio. Entonces se puede representar de forma única como $ a_1| u_1 \rangle + a_2| u_2 \rangle+\cdots + a_n| u_n \rangle,  $. Pero obsérvese que $ |u_i \rangle \langle u_i|| x \rangle = \langle u_i| x \rangle |u_i \rangle $. Pero, $ \langle u_i| x \rangle |u_i \rangle = \langle u_i| (a_1| u_1 \rangle + a_2| u_2 \rangle+\cdots + a_n| u_n \rangle) \rangle |u_i \rangle = (a_1 \langle u_i|u_1 \rangle  + a_2 \langle u_i|u_2 \rangle + \cdots + a_n \langle u_i|u_n \rangle) | u_i \rangle $, pero como la base es ortonormal, todos los términos exceptuando $ a_i\langle u_i | u_i \rangle = a_i$ desapareceran y obtenemos que $ \langle u_i| x \rangle |u_i \rangle = a_i |u_i \rangle$.

Por lo tanto, $ A = (\sum_{i=1}^n | u_i \rangle \langle u_1 |)|x \rangle = \sum_{i=1}^n \langle u_i|x \rangle | u_i \rangle  = \sum_{i=1}^n a_i | u_i \rangle = |x \rangle $.

\item $ |\psi \rangle \langle \phi |$ es tal que envia $ |x \rangle  $ a $ \langle \phi | x \rangle |\psi \rangle $. Tenemos que

$ \langle \phi | 1 \rangle |\psi \rangle = \langle (e|1\rangle + f|2\rangle +g|3\rangle)| 1 \rangle |\psi \rangle  = (\overline{e}\langle 1|1 \rangle + \overline{f}\langle 2|1 \rangle + \overline{g}\langle 3|1 \rangle )|\psi \rangle = \overline{e}|\psi \rangle = \overline{e}a|1 \rangle + \overline{e}b|2 \rangle + \overline{e}c|3 \rangle$

Similarmente,

$ \langle \phi | 2 \rangle |\psi \rangle = \langle (e|1\rangle + f|2\rangle +g|3\rangle)| 2 \rangle |\psi \rangle  = (\overline{e}\langle 1|2 \rangle + \overline{f}\langle 2|2 \rangle + \overline{g}\langle 3|2 \rangle )|\psi \rangle = \overline{f}|\psi \rangle = \overline{f}a|1 \rangle + \overline{f}b|2 \rangle + \overline{f}c|3 \rangle$

Y por último,

$ \langle \phi | 3 \rangle |\psi \rangle = \langle (e|1\rangle + f|2\rangle +g|3\rangle)| 3 \rangle |\psi \rangle  = (\overline{e}\langle 1|3 \rangle + \overline{f}\langle 2|3 \rangle + \overline{g}\langle 3|3 \rangle )|\psi \rangle = \overline{g}|\psi \rangle = \overline{g}a|1 \rangle + \overline{g}b|2 \rangle + \overline{f}g|3 \rangle$

Luego la representación matricial de $ |\psi \rangle \langle \phi | $ es 

$$ \begin{pmatrix}
\overline{e}a & \overline{f}a & \overline{g}a
\\\overline{e}b & \overline{f}b & \overline{g}b
\\\overline{e}c & \overline{f}c & \overline{g}c
\end{pmatrix} $$

\item \begin{enumerate}
\item Tenemos que $ \langle a | (\alpha A)^\dagger | b \rangle = \overline{\langle b | \alpha A | a \rangle} = \overline{\alpha} \overline{\langle b | A | a \rangle} = \overline{\alpha}{\langle a | A^\dagger | b \rangle} ={\langle a | \overline{\alpha} A^\dagger | b \rangle} $.

\item 
\end{enumerate}

\item Calcule el adjunto de $ 2\ket{0}\bra{1}-i\ket{1}\bra{0} $

La matriz asociada a este elemento es 

$$ \begin{pmatrix}
0 & 2
\\ -i & 0
\end{pmatrix} $$

Por el punto anterior sabemos que $ A^\dagger $ es la transpuesta de la conjugada de $ A $ luego la matriz adjunta es
 
$$ \begin{pmatrix}
0 & i
\\ 2 & 0
\end{pmatrix} $$ 

\item $ A = \begin{pmatrix}
0 & 0 & i
\\0 & 1 & 0
\\ -i & 0 & 0
\end{pmatrix}
 $
 
 Calculamos valores y vectores propios
 
 $ |A-\lambda Id| = \begin{vmatrix}
-\lambda & 0 & i
\\ 0 & 1 - \lambda & 0
\\ -i & 0 & -\lambda
\end{vmatrix} =
-\lambda\begin{vmatrix}
1-\lambda & 0
\\0 & -\lambda
\end{vmatrix} + i \begin{vmatrix}
0 & 1-\lambda
\\-i & 0
\end{vmatrix} =$

$ - \lambda(1-\lambda)(- \lambda)+i(-1)(1-\lambda)(-i) = -\lambda^3+\lambda^2+\lambda-1 = -(\lambda+1)(\lambda^2-2\lambda+1)=-(\lambda+1)(\lambda-1)^2$

El vector propio asociado a -1 lo obtenemos del kernel de
$ \begin{pmatrix} 
1 & 0 & i
\\ 0 & 2 & 0
\\ -i & 0 & 1
\end{pmatrix}  $

Calculando obtenemos que este kernel es generado por el vector normalizado $ \ket{1}=\frac{1}{\sqrt{2}}\begin{pmatrix}
1
\\0
\\-i
\end{pmatrix} $

Por otra parte los valores propios asociados al valor 1 se obtienen del kernel de $ \begin{pmatrix}
-1 & 0 & i
\\0 &0 & 0
\\-i & 0 & -1
\end{pmatrix} $

Calculando obtenemos que este espacio es generado por los vectores normalizados $ \ket{2} = \frac{1}{\sqrt{2}}\begin{pmatrix}
1  \\ 0 \\ i
\end{pmatrix} $ y  $ \ket{3}=\begin{pmatrix}
0  \\ 1 \\ 0
\end{pmatrix} $ 

Estos vectores forman una base ortonormal por lo que podemos escribir $ A = -1\ket{1}\bra{1}+\ket{2}\bra{2}+\ket{3}\bra{3}$.

\item Las matrices de Pauli son

$ \sigma_x = \begin{pmatrix}
0 & 1
\\1 & 0
\end{pmatrix}$ 
$ Y = \begin{pmatrix}
0 & -i
\\i & 0
\end{pmatrix}$ 
$ Z = \begin{pmatrix}
1 & 0
\\0 & -1
\end{pmatrix}$

diagonalize las matrices de Pauli.

El polinomio característico de $ \sigma_x $ es $ \lambda^2-1 = (\lambda-1)(\lambda+1)$. El vector propio asociado a $ 1 $ es $ \ket{+}= \frac{1}{\sqrt{2}}\begin{pmatrix}
1 \\ 1
\end{pmatrix} $ y el vector propio asociado a -1 es $ \ket{-}= \frac{1}{\sqrt{2}}\begin{pmatrix}
1 \\ -1
\end{pmatrix}$ 

El polinomio característico de $ Y $ es también $ \lambda^2-1 = (\lambda-1)(\lambda+1)$. El vector propio asociado a $ 1 $ es $ \ket{+}= \frac{1}{\sqrt{2}}\begin{pmatrix}
1 \\ 1
\end{pmatrix} $ y el vector propio asociado a -1 es $ \ket{-}= \frac{1}{\sqrt{2}}\begin{pmatrix}
1 \\ -1
\end{pmatrix}$ 

\item Muestre que $ Tr(A \ket{\phi}\bra{\psi})= \braket{\psi|A|\phi} $
$ Tr(A)= \sum_{i=1}^n a_ii = \sum_{i=1}^n \braket{u_i|A|u_i} $.

Por otra parte, 
\begin{eqnarray}
Tr(A|\phi \times \psi|) &=& \sum_{i=1}^n \braket{u_i|A|\phi \times \psi | u_i} \nonumber
\\ &=& \sum_{i=1}^n \braket{ \psi | u_i \times u_i|A|\phi} \nonumber
\\ &=& \braket{ \psi | \sum_{i=1}^n u_i \times u_i|A|\phi} \nonumber
\end{eqnarray}
Pero vimos que $ \sum_{i=1}^n u_i \times u_i$ es la identidad luego esto es igual a $ \braket{\psi|A|\phi} $.

\item

\item \begin{enumerate}

\item \begin{eqnarray} [\sigma_x, \sigma_y ] &=& \sigma_x\sigma_y - \sigma_y\sigma_x \nonumber
\\ &=& \begin{pmatrix} 0 & 1 \\ 1 & 0 \end{pmatrix}\begin{pmatrix} 0 & i \\ -i & 0 \end{pmatrix}-\begin{pmatrix} 0 & i \\ -i & 0 \end{pmatrix}\begin{pmatrix} 0 & 1 \\ 1 & 0 \end{pmatrix} \nonumber
\\ &=& \begin{pmatrix} -i & 0 \\ 0 & i \end{pmatrix}-\begin{pmatrix} i & 0 \\ 0 & -i \end{pmatrix} \nonumber
\\ &=& \begin{pmatrix} -2i & 0 \\ 0 & 2i \end{pmatrix}= -2i\sigma_z\nonumber
\end{eqnarray}
\item \begin{eqnarray} [\sigma_y, \sigma_z ] &=& \sigma_y\sigma_z - \sigma_z\sigma_y \nonumber
\\ &=& \begin{pmatrix} 0 & i \\ -i & 0 \end{pmatrix}\begin{pmatrix} 1 & 0 \\ 0 & -1 \end{pmatrix}-\begin{pmatrix} 1 & 0 \\ 0 & -1 \end{pmatrix}\begin{pmatrix} 0 & i \\ -i & 0 \end{pmatrix} \nonumber
\\ &=& \begin{pmatrix} 0 & -i \\ -i & 0 \end{pmatrix}-\begin{pmatrix} 0 & i \\ i & 0 \end{pmatrix} \nonumber
\\ &=& \begin{pmatrix} 0 & -2i \\ -2i & 0 \end{pmatrix}= -2i\sigma_x\nonumber
\end{eqnarray}
\item \begin{eqnarray} [\sigma_z, \sigma_x ] &=& \sigma_z\sigma_x - \sigma_x\sigma_z \nonumber
\\ &=& \begin{pmatrix} 1 & 0 \\ 0 & -1 \end{pmatrix}\begin{pmatrix} 0 & 1 \\ 1 & 0 \end{pmatrix}-\begin{pmatrix} 0 & 1 \\ 1 & 0 \end{pmatrix}\begin{pmatrix} 1 & 0 \\ 0 & -1 \end{pmatrix} \nonumber
\\ &=& \begin{pmatrix} 0 & 1 \\ -1 & 0 \end{pmatrix}-\begin{pmatrix} 0 & -1 \\ 1 & 0 \end{pmatrix} \nonumber
\\ &=& \begin{pmatrix} 0 & 2 \\ -2 & 0 \end{pmatrix}= -2i\sigma_y\nonumber
\end{eqnarray}
\end{enumerate}
$ \ket{+}\bra{0}+\ket{-}\bra{1} =(1/\sqrt{2}\ket{0}+1/\sqrt{2}\ket{1})\bra{0} + (1/\sqrt{2}\ket{0}-1/\sqrt{2}\ket{1})\bra{1} $

$ = 1/\sqrt{2}\ket{0}\bra{0}+1/\sqrt{2}\ket{1}\bra{0}+1/\sqrt{2}\ket{0}\bra{1}-1/\sqrt{2}\ket{1}\bra{1}  $ 

$ = 1/\sqrt{2}(\ket{0}\bra{0}+\ket{1}\bra{0}+\ket{0}\bra{1}-\ket{1}\bra{1})  $

$ = 1/\sqrt{2}(\begin{pmatrix}
1 & 0 
\\ 0 & 0
\end{pmatrix}+\begin{pmatrix}
 0 & 0 
\\ 1 & 0
\end{pmatrix}+\begin{pmatrix}
 0 & 1 
\\ 0 & 0
\end{pmatrix}-\begin{pmatrix}
 0 & 0 
\\ 0 & -1
\end{pmatrix})  $

$ = \frac{1}{\sqrt{2}}\begin{pmatrix}
 1 & 1 
\\ 1 & -1
\end{pmatrix} $

\item \begin{enumerate}
\item $ \sigma_x \otimes \sigma_y = \begin{pmatrix}
0 & 0 & 0 & -i 
\\ 0 & 0 & i & 0
\\ 0 & -i & 0 & 0
\\ i & 0 & 0 & 0
\end{pmatrix}$

\item  $ \sigma_y \otimes \sigma_x = \begin{pmatrix}
0 & 0 & 0 & -i 
\\ 0 & 0 & -i & 0
\\ 0 & i & 0 & 0
\\ i & 0 & 0 & 0
\end{pmatrix}$

\item $ \sigma_z \otimes \sigma_x = \begin{pmatrix}
0 & 1 & 0 & 0 
\\ 1 & 0 & 0 & 0
\\ 0 & 0 & 0 & -1
\\ 0 & 0 & -1 & 0
\end{pmatrix}$ 

\item $ \sigma_x \otimes \sigma_y \ket{\phi} = \begin{pmatrix}
0 & 0 & 0 & -i 
\\ 0 & 0 & -i & 0
\\ 0 & i & 0 & 0
\\ i & 0 & 0 & 0
\end{pmatrix} \ket{\phi}=
\begin{pmatrix}
0 & 0 & 0 & -i 
\\ 0 & 0 & -i & 0
\\ 0 & i & 0 & 0
\\ i & 0 & 0 & 0
\end{pmatrix} \begin{pmatrix} 0 \\ \frac{1}{\sqrt{2}} \\ -\frac{1}{\sqrt{2}} \\ 0\end{pmatrix} = 
\begin{pmatrix} 0 \\ \frac{i}{\sqrt{2}} \\ \frac{i}{\sqrt{2}} \\ 0\end{pmatrix}$ 
\end{enumerate}
\item El enunciado es falso.

Tomese por ejemplo la base ortonormal encontrada en el punto 3.

$ \ket{u_1} = \begin{pmatrix}
i/\sqrt{2}
\\ 1/\sqrt{2}
\end{pmatrix} $ 
 $ \ket{u_2} = \begin{pmatrix}
 (i-1)/2
 \\ (i+1)/2
 \end{pmatrix}$ 

Calculando la expresión $ \ket{u_1} \otimes \ket{u_1} + \ket{u_2} \otimes \ket{u_2} = \begin{pmatrix} -1/2 \\ i/2 \\ i/2 \\ 1/2 \end{pmatrix} + \begin{pmatrix} -i/2 \\ -1/2 \\ -1/2 \\ i/2 \end{pmatrix} = \begin{pmatrix} -1/2-i/2 \\ i/2-1/2 \\ i/2-1/2 \\ 1/2+i/2 \end{pmatrix} $

Vemos que esto es diferente a $ \ket{00}+\ket{11} = \begin{pmatrix} 1 \\ 0 \\ 0 \\ 1 \end{pmatrix}$

Creemos que la expresión correcta hubiera sido tomando el producto tensorial entre los vectores y sus conjugados.

\end{enumerate}

\begin{enumerate}
\item Veamos que $ \{N_{l,m}\} $ es una familia de operadores de medición.

En efecto,

\begin{equation}
\sum_{l,m}N_{l,m}^\dagger N_{l,m} = \sum_{l,m}(M_mL_l)^\dagger M_mL_l = \sum_{l,m}L_l^\dagger M_m^\dagger M_mL_l \nonumber = \sum_{l}L_l^\dagger \sum_{m}(M_m^\dagger M_m)L_l
 \end{equation} 
Pero la suma interna es igual a la identidad por nuestra suposición que $ \{M_m\} $ es una familia de operadores de medición. Luego obtenemos que $ \sum_{l,m}N_{l,m}^\dagger N_{l,m} =  \sum_{l}L_l^\dagger (I)L_l =  \sum_{l}L_l^\dagger L_l = I$

y esto ultimo también es igual a la identidad por nuestra suposición que $ \{L_l\} $ también es una familia de operadores de medición.

\item En un punto anterior se mostró que $ \ket{u_i}\bra{u_i} $ es el operador de proyección sobre la coordenada generada por el vector $ u_1 $. Las propiedaddes de las proyecciones es que son auto-adjuntas e idempotentes. Por lo tanto, usando primero que es autoadjunta y luego que es idempotente tenemos que

\begin{equation}
\ket{u_i}\bra{u_i}^\dagger\ket{u_i}\bra{u_i} = \ket{u_i}\bra{u_i}\ket{u_i}\bra{u_i} = \ket{u_i}\bra{u_i} \nonumber
\end{equation}

Luego $ \sum_{i = 0}^n \ket{u_i}\bra{u_i}^\dagger\ket{u_i}\bra{u_i} = \sum_{i = 0}^n \ket{u_i}\bra{u_i} $ y en un punto anterior demostramos que esto es igual a la Identidad. Por lo tanto, esto forma una familia de operadores de medición.

\item $ \{P_v, I-P_v\} $ es una familia de operadores de medición.

En efecto tomemos la expresión $ P_v^\dagger P_v+(I-P_v)^\dagger(I-P_v) $. Tenemos que $ (A+B)^\dagger = A^\dagger+ B^\dagger $ y que $ (\alpha A)^\dagger = \overline{\alpha}A^\dagger $. Además $ I = I^\dagger $ es unitaria y autoadjunta. Luego la expresión es equivalente a $ P_v^\dagger P_v +I-P_v- P_v^\dagger + P_v^\dagger P_v $. Pero en el punto anterior vimos que las proyecciones son autoadjuntas e idempotentes.

Luego $ P_v^\dagger P_v = P_v $ y $ P_v^\dagger = P_v $. Por lo tanto, la expresión se reduce a $ P_v+I-P_v-P_v+P_v = I $.

\item \begin{itemize}
\item El primer bit sea 0. El operador asociado es
$ \ket{0}\bra{0} \otimes \overbrace{I \otimes\cdots\otimes  I}^\text{n-1-veces} $

\item Los primeros $ m $-bits sean 0. El operador asociado es
$ \overbrace{\ket{0}\bra{0} \otimes \cdots \otimes \ket{0}\bra{0}}^\text{m-veces} \otimes \overbrace{I \otimes\cdots\otimes  I}^\text{n-m-veces} $

\item El primer bit sea 0 y el segundo sea 1. El operador asociado es
$  \ket{0}\bra{0} \otimes \overbrace{I \otimes\cdots\otimes  I}^\text{n-2-veces} \otimes \ket{1}\bra{1}  $

\item Todos los bits sean 1.

$  \overbrace{\ket{1}\bra{1} \otimes\cdots\otimes  \ket{1}\bra{1}}^\text{n-veces} $

\end{itemize}

\item En puntos anteriores ya vimos que $ P_i $ es un operador autoadjunto e idempotente. El teorema espectral me da una descomposición del espacio como la suma directa de los espacios asociados a cada valor propio.

Tenemos que ver que $ \sum_{\lambda \in \text{Spec}(A)} P_\lambda^\dagger P_\lambda = I $. Pero ya vimos que $ P_\lambda^\dagger P_\lambda = P_\lambda $.

Luego $ \sum_{\lambda \in \text{Spec}(A)} P_\lambda^\dagger P_\lambda = \sum_{\lambda,i} \ket{v_{\lambda,i}}\bra{v_{\lambda,i}} = I $ por el punto 2.

\item \begin{enumerate}
 \item La probabilidad que $ \ket{\psi} $ este en $ \ket{01} $ es $ 3/8 $.
 
 \item La probabilidad que $ \ket{\psi} $ tenga el primer q-bit 0 es $ 1/8 + 3/8 = 1/2 $. La probabilidad que tenga el primer q-bit 1 es $ 1/4+1/4 = 1/2 $
 
 \item Después de medir en el liter a) el estado resultante debe ser $ \ket{\psi'}=\ket{01} $.
 
 En el literal b) en el primer caso el estado resultante sería
 
 \begin{equation}
 \ket{\psi'}=\frac{1/\sqrt{8}\ket{00} + \sqrt{3/8}\ket{01}}{1/\sqrt{2}}=1/2\ket{00}+\sqrt{3}/2\ket{01} \nonumber
 \end{equation}
 
 Y en el segundo caso sería
\begin{equation}
\ket{\psi'}=\frac{1/2\ket{10} + 1/2\ket{11}}{1/\sqrt{2}}=1/\sqrt{2}\ket{10}+1/\sqrt{2}\ket{11} \nonumber
\end{equation}
\end{enumerate}

\item 

\item 

\item Primero para ver condiciones necesarias tome $ \begin{pmatrix} x \\ y \end{pmatrix} $ y $ \begin{pmatrix} z \\ w \end{pmatrix} $. Vemos que el producto tensorial es igual a
$ \begin{pmatrix}
xz
\\xw
\\yz
\\yw
\end{pmatrix} $

Encontramos el criterio de Peres-Horodecki que nos indica que una condición necesaria (y suficiente para dimensión 2 * 2). El criterio se trata de evaluar la matriz $ \ket{\phi}\bra{\phi} $, esta puede verse como $ \sum_{k,l\in B}\ket{k}\bra{l} $ donde $B$ son las proyecciones sobre los vectores de la base canónica de $ \mathbb{C}^2 \otimes \mathbb{C}^2$ es decir $ a\ket{00} $, $ b\ket{01} $, $ c\ket{10} $ y $ d\ket{11} $. A su vez $ \ket{ab}\bra{ab}= \ket{a}\bra{a} \otimes \ket{b}\bra{b}$. Luego la suma queda como $ \sum_{j,k\in \{0,1\}} \ket{j}\bra{j} \otimes \ket{k}\bra{k}$.

Luego se calcula la matriz $ A^{T_B} $ es igual a $ \sum_{j,k\in \{0,1\}} \ket{j}\bra{j} \otimes (\ket{k}\bra{k})^T$. El criterio dice que $ \phi $ es entrelazado si $ A^{T_B} $ tiene algún valor negativo.
\item  


\end{enumerate}
\end{document}