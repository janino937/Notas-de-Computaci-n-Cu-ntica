\documentclass[letter,twoside,12pt]{article}
\usepackage{lmodern}
\usepackage[T1]{fontenc}
\usepackage[spanish]{babel}
\usepackage[utf8]{inputenc}
\usepackage{amsmath}
\usepackage{amssymb}
\usepackage{amsthm}
\usepackage{fullpage}
\usepackage{latexsym}
\usepackage{enumerate}
\usepackage{enumitem}
\PassOptionsToPackage{hyphens}{url}\usepackage{hyperref}
\title{Notas de la clase de computación cuántica}
\newtheorem{lemma}{Lema}
\author{Jonathan Andrés Niño Cortés}
\begin{document}
\maketitle

Una máquina de Turing $ M $ trabaja en timepo $ T(n) $ si para cada entrada de longitud $ n $, se necesitan a lo más $ T(n) $ pasos para llevar a cabo el cálculo.

Una función $ f(n) $ es crecimiento polinomial si $ f(n)  \leq cn^\alpha$ para alguna constante $ c > 0 $ y $ \alpha \in \mathbb{N} $. $ n>>0 $.

\textbf{Definición:} Una función $ f:\mathbb{B}^{*} \to \mathbb{B}^{*} $ es calculable en tiempo polinomial

Si $ \exists $ MT que calcula $ F $ en tiempo $ T(n) = Poly(n)$.

\begin{equation}
P = \{F:\mathbb{B}^*\to\mathbb{B}^*| F \text{ es calculable en tiempo polinomial} \}
\end{equation}

\textbf{Definición:} Se dice que MT trabaja en espacio $ S(n) $ si visita a lo más $ S(n) $ "células" de la cinta de la máquina

\begin{equation}
PSpace = \{F:\mathbb{B}^*\to\mathbb{B}^*| F \text{ es calculable en espacio polinomial} \}
\end{equation}

\textbf{Conjetura:} $ P \subset PSpace $.

Un predicado es una función $ P: \mathbb{B}^* \to \mathbb{B} $.

\textbf{Ej:} Una función que me indique si los números son primos.

$ D:\mathbb{B}^* \to \mathbb{B}$ tal que $ D(n) $ es 0 si $ n $ es primo y $ 1 $ si $ n $ no es primo.

Un predicado en dos variables $ R: \mathbb{B}^* \times \mathbb{B}^* \to \mathbb{B}$ es calculable en timepo polinomial si existe MT que calcule R en tiempo Poly($ |x|,|y| $), es decir, la máquina de Turing usa a lo más Poly($ |x|+|y $) tiempo.

Los problemas NP son tales que uno puede dar una prueba que justifique el resultado dado por las funciones NP.

\textbf{Def:} Un predicado $ L:\mathbb{B}^* \to \mathbb{B}$ esta en la clase NP si se puede representar como

\begin{equation}
L(x) = \exists y (|y| < q(|x|) \wedge R(x,y))
\end{equation}

donde $ q $ es un polinomio y $ R $ es un predicado en dos variables calculable en tiempo polinomial.

Al menos una respuesta se puede dar por certificado. 

Tenemos que $ P \subseteq NP $. Pero áun no sabemos si el converso es cierto o falso.

\textbf{Computación Cuántica}. La idea de la computación es encontrar un algoritmo que calcule una función.

Podemos suponer que una función $ F: \mathbb{B}* \to \mathbb{B}^*$ la podemos partir en partes más pequeñas.

\begin{equation}
n \in \mathbb{N} F_n: \mathbb{B}^n \to \mathbb{B}^{f(n)} .
\end{equation}

Obsérvese que el número de funciones $ B \to B $ es $ 2^{2^n} $ así que la complejidad de encontrar estas funciones es muy grande.

Vamos a probar que los conectores lógicos $ \wedge, \vee , \neg$. pueden realizar cualquier función calculable.

\begin{equation}
L = \{a \in \mathbb{B}^n|f(a)=1 \}
\end{equation}

\begin{equation}
f_{(x)}^{(a)}= \text {1 si } x = a \text{ 0 si } x  \not = a
\end{equation}

\begin{equation}
f(x)= \bigvee_{a \in L} \chi^{a}(x)
\end{equation}

\begin{equation}
\chi^{(a)}: \mathbb{B}^s \to \mathbb{B}, a =11111
\end{equation}

\begin{equation}
\chi^{(a)}(x_1,x_2,x_3,x_4,x_5)=x_1 \wedge x_1 \wedge x_2 \wedge x_3 \wedge x_4 \wedge x_5
\end{equation}


\begin{equation}
\chi^{(b)}(x_1,x_2,x_3,x_4,x_5)=\neg x_1 \wedge x_1 \wedge x_2 \wedge x_3 \wedge x_4 \wedge x_5
\end{equation}


\textbf{Teorema:} Toda función booleana se puede construir usando $ \wedge, \vee, \neg $.

\textbf{Def:} Un circuito booleano con base $ \{f_1,\cdots, f_k \}  (f_i: \mathbb{B}^{n_i} \to \mathbb{B}^{m_i})$ es una sucesión finita de funciones de la forma 
\begin{equation}
l_n=y_1^n \times y_2^n \times \cdots \times y_{s_n}^{n}
\end{equation}

donde los $ y_i^n $ son elementos de $ B $ o son $ id_{B} $ o si $ n=1 $ son funciones constantes $ 1: \mathbb{B} \to \mathbb{B} $, $ 0: \mathbb{B} \to \mathbb{B} $. que además se pueden componer entre si.
\end{document}